%\documentstyle[amssymb,11pt,emlines,twoside,russian]{article}
%\documentclass[12pt,draft]{amsproc}
%\documentclass[a4paper,11pt,twoside]{amsproc}
\documentclass[a4paper,11pt,twoside]{article}
\usepackage{amsmath,amssymb,amsthm,eucal,graphics}
\usepackage[utf8]{inputenc}
\usepackage[russian]{babel}
\usepackage{graphicx}
\usepackage{amsfonts,amssymb,eucal}
\usepackage{amsbsy}
\usepackage{latexsym}
\usepackage{color}
\usepackage{hyperref}

%\usepackage{size11,emlines}

%\setlength{\textwidth}{38.72pc}
\setlength{\textwidth}{130mm}
%\setlength{\textheight}{52.2pc}
\setlength{\textheight}{225mm}

%\setlength{\oddsidemargin}{5mm}
\setlength{\evensidemargin}{\oddsidemargin}
\renewcommand{\baselinestretch}{1.40}
\addtolength{\topmargin}{-5mm}
\addtolength{\textheight}{-12mm}
\parindent=2.5em

\headheight 15pt
\headsep 18pt

\makeatletter
\gdef\firstpage{1}

%\def\footnoterule{\ifnum\thepage=\firstpage \firstrule \else
%\standard\fi}
%\def\firstrule{\kern -3pt \hrule width \textwidth \kern 2.6pt}
%\def\standard{\kern -3pt \hrule width 2truein \kern 2.6pt}
\def\year{2015}
\def\frsthdr{Алгебра и логика, 50, № 1 (2011), 1---???}

\def\firstpageone{0\thepage}
\def\firstpagetwo{00\thepage}
\def\firstpagethree{000\thepage}
\def\firstpagemark{\ifnum\firstpage <10  \firstpageone
\else\ifnum\firstpage<100 \firstpagetwo \else \ifnum\firstpage
<1000 \firstpagethree \else \firstpageone\fi\fi\fi}

\def\footline{\ifnum\thepage=\firstpage \footlineone
%\gdef\firstpage{0}
\else\footlinetwo\fi}
\def\footlineone{\noindent \footnotesize \sf \copyright\ \ Сибиpский фонд
алгебpы и логики,\ \year  \hspace{\fill}  \hbox{}}
\def\footlinetwo{}

\def\titles{{Универсальные инварианты для классов абелевых групп}}
\title{\titles\footnote{Исследование выполнено за счет гранта Российского научного фонда (проект \No 14-11-00085)}}
%\def\translation#1{\gdef\firstpage{\thepage}\footnote{{\small\rm
%\hglue-.6cm #1}}}
\def\authors{{А.А. Мищенко, В.Н. Ремесленников, А.В. Трейер.}}
\author{\authors}

\def\oddhedr{\ifnum\thepage=\firstpage \firsthdr \else \odhdr \fi}
\def\firsthdr{\hspace{\fill} \sl \frsthdr \hspace{\fill}\hbox{}}
\def\odhdr{\hspace{\fill}\sl\rightmark \titles \hspace{\fill}
\rm \thepage}

\def\evnhedr{\ifnum\thepage=\firstpage \firsthdr \else \evhdr \fi}
\def\evhdr{\noindent \rm \thepage\hspace*{\fill} \sl\leftmark
\authors \hspace*{\fill}\hbox{}}

\def\ps@newpstyle{\def\@oddhead{
\hspace{-0.65em} \vbox{\oddhedr\vskip 1mm \hrule width
\textwidth}
}
%\def\@oddfoot{}
\def\@evenhead{
\hspace{-0.65em} \vbox{\evnhedr\vskip 1mm \hrule width
\textwidth}
}\textsc{}
\def\@oddfoot{\footline}
\def\@evenfoot{\@oddfoot}
}

% \def\thefootnote{}
\renewcommand*{\thefootnote}{\fnsymbol{footnote}}
\raggedbottom

\let\goth\mathfrak

\def\refer
{
%\section*{References}
\begin{small}
\begin{enumerate}
%\listparindent=-2em
%\itemindent=-1em
\leftmargin=0pt
\rightmargin=0pt
\itemsep=0pt
\parsep=0pt
}

\def\endref{\end{enumerate}\end{small} }


\newtheorem{theorem}{Теорема}[section]
\newtheorem{lemma}{Лемма}[section]
\newtheorem{proposition}{Предложение}[section]
\newtheorem{statement}{Утверждение}[section]
\newtheorem{corollary}{Следствие}[section]
\newtheorem{definition}{Определение}[section]

\def\note#1{\marginpar{\textcolor{red}{#1}}}
\def\proof{{\noindent{\bf Доказательство.}} }
\def\A{{\mathfrak{A}}}
\def\K{{\mathcal{K}}}
\def\U{{\mathcal{U}}}
\def\P{{\mathcal{P}}}
\def\F{{\mathcal{F}}}
\def\S{{\mathcal{S}}}
\def\L{{\mathcal{L}}}
\def\C{{\mathcal{C}}}
\def\Z{{\mathbb{Z}}}
\def\N{{\mathbb{N}}}
\def\Q{{\mathbb{Q}}}
\def\Th{{\mathrm{Th}}}
\def\Tha{{\mathrm{Th}_\forall}}
\def\The{{\mathrm{Th}_\exist}}
\def\CG{{\mathrm{CGr}}}
\def\ui{{\mathrm{UI}}}




\begin{document}

\maketitle

\tableofcontents


\setcounter{page}{\firstpage}
\pagestyle{newpstyle}

\Russian
\sloppy
\rm

\section{Введение}

Проблема элементарной эквивалентности абелевых групп решена в статье В.Шмелевой \cite{Szm}. В этой работе доказан следующий результат: абелевы группы $A$ и $B$ элементарно эквивалентны тогда и только тогда, когда значения элементарных инвариантов группы $A$ совпадают со значениями элементарных инвариантов для группы $B$. Определение элементарных инвариантов дано в параграфе \ref{sec:UnivInvariants}.

Цель данной работы следующая. Мы заменяем элементарную теорию $\Th(A)$ для абелевой группы $A$ на универсальную теорию этой группы $\Tha(A)$ и вводим универсальный инвариант $\ui(A)$ для группы $A$ как последовательность 
$$\ui(A) = (\ui_0(A), \ui_2(A), \ui_3(A), \ui_5(A), \ldots, \ui_{p_n}(A), \ldots ),$$
где $\ui_{p_n}(A)$ -- вектор составленный из значений элементарных инвариантов, а $p_n$ -- простое число, и доказываем аналог теоремы Шмелевой об универсальной эквивалентности абелевых групп (теорема \ref{th:UnivEquivOfGroups}).

Далее мы переходим к классам абелевых групп и определяем универсальные инварианты $\ui(\K)$ для класса $\K$, и доказываем аналог теоремы Шмелевой для классов абелевых групп (теорема \ref{th:UnivClasses}).

Кроме того, в параграфе \ref{sec:CannonicalGroups} мы вводим множество $\CG$ -- канонических абелевых групп. Теорема \ref{th:CannonicalGroups} утверждает, что множество $\CG$ есть множество всех представителей для универсальных классов абелевых групп по одной группе для каждого класса. 

\section{Предварительные сведения}

\subsection{Связь между универсальной эквивалентностью и дискриминируемостью}

В данном параграфе мы введем два основных понятия, которые мы будем существенно использовать в этой статье. Первое понятие -- это понятие конечной диаграммы, а второе -- понятие локальной дискриминируемости групп класса $\K_1$ группами класса $\K_2$.

Напомним определения из теории моделей. Пусть $L$ -- групповой язык, $X = \{x_1, \ldots, x_n, \ldots\}$ -- множество букв. Атомарной формулой называется равенство или неравенство термов от переменных из множества $X$. Конечное множество атомарных формул $S$ будем называть конечной диаграммой. Будем говорить, что конечная диаграмма $S$ реализуется в модели $M$, если существуют элементы $m_1, \ldots, m_n \in M$ на которых выполняются все атомарные формулы из $S$. Это утверждение на языке формул означает, что в моделе $M$ выполнена экзистенциальная формула $\varphi_s = \exists x_1, \ldots, x_n \ \bigwedge\limits_{\varphi \in S} \varphi(x_1, \ldots, x_n)$. Отрицание $\neg\varphi_s$ формулы $\varphi_s$ является универсальной формулой. Если универсальная формула $\neg\varphi_s$ выполнена на моделе $M$, то в ней нет набора из $n$ элементов, удовлетворяющих формуле $\varphi_s$. Другими словами все универсальные формулы, которые выполнены на моделе $M$ задают набор конечных диаграм, которые не реализуются в модели $M$. Множество всех конечных диаграмм с точностью до изоморфизма, которые реализуются в моделе $M$ обозначим $FD(M)$. 
\begin{definition}
Будем говорить, что две группы $A$ и $B$ универсально эквивалентны (символически $A \equiv_\forall B$) если их универсальные теории совпадают. Под универсальной теорией группы $A$ мы понимаем множество всех универсальных формул истинных на группе $A$.
\end{definition}
Справедлива следующая лемма, доказательство которой следует из рассуждений выше.

\begin{lemma}
Две группы $A$ и $B$ универсально эквивалентны тогда и только тогда, когда $FD(A) = FD(B)$.
\end{lemma}

Для группы $G$ обозначим $FS(G)$ множество классов изоморфизмов конечных подгрупп $G$.

\begin{lemma}\label{lemma:UnivEquivFS}
Две периодические абелевы группы $A$ и $B$ универсально эквивалентны тогда и только тогда, когда $FS(A) = FS(B)$.
\end{lemma}
\proof Из определения конечной диаграммы следует, что если конечная диаграмма реализуется на группе $A$, то она реализуется на конечно порожденной подгруппе в $A$. Но так как группы $A$ и $B$ периодические и абелевы, то любая конечно порожден подгруппа в них является конечной. Кроме того, каждую конечную подгруппу можно задать конечной диаграммой. Из этого непосредственно следует доказательство леммы. $\square$



Пусть $\K$ -- класс абелевых групп. Обозначим через $FD(\K)$ множество всех конечных диаграмм с точностью до изоморфизма, которые реализуются в некоторой модели $M$ из класса $\K$. Ясно, что отрицание формул из $FD(\K)$ есть универсальная теория класса $\K$.

Сформулируем второе основное понятие для классов $\K_1$ и $\K_2$ -- понятие локальной дискриминируемости класса $\K_1$ классом $\K_2$.
\begin{definition}
Класс $\K_1$ локально дискриминируется классом $\K_2$, если для любой системы $A$ из $\K_1$ и любой конечной диаграммы $S \in FD(A)$ существует система $B \in \K_2$ и гомоморфизм $\varphi: A \rightarrow B$ такие, что если конечная диаграмма $S$ реализуется на элементах $a_1, \ldots, a_n \in A$, то эта конечная диаграмма реализуется в системе $B$ на элементах $\varphi(a_1), \ldots, \varphi(a_n)$.
\end{definition}

Это понятие тесно связано с понятием универсальной эквивалентности двух групп. В этой связи, напомним один результат из статьи Э.Ю.~Данияровой, А.Г.~Мясникова и В.Н.~Ремесленникова \cite{DMR1}. Для удобства читателя, сформулируем нужные нам пункты данной теоремы в форме следующей леммы.

\begin{lemma}\label{lemma:UnivEquiv}
Пусть $A$ -- нетерова по уравнениям алгебра языка $L$. Тогда для конечно порожденной алгебры $B$ языка $L$ следующие условия эквивалентны:
\begin{enumerate}
\item $\mathrm{Th}_{\forall}(A) \subseteq \mathrm{Th}_{\forall}(B)$;
\item $B$ дискриминируется $A$;
\end{enumerate}
\end{lemma}

Понятие нетеровой алгебры вводится в той же статье трех авторов, и для группы $A$ звучит следующим образом. Группа $A$ нетерова по уравнениям тогда и только тогда, когда для любой бесконечной системы уравнений $S$ от конечного числа неизвестных, существует её конечная подсистема $S_0$ такая что, множество решений системы $S$ совпадает с множеством решений системы $S_0$. В работе \cite{DMR2} доказано, что любая абелева группа является нетеровой группой по уравнениям.

Результаты этой леммы мы расширяем на классы абелевых групп.

\begin{lemma}\label{lemma:UnivEquivForClass}
Пусть $\K_1$ и $\K_2$ -- два класса абелевых групп. Тогда следующие утверждения эквивалентны:
\begin{enumerate}
\item $\Tha(\K_1) \subseteq \Tha(\K_2)$;
\item Класс $\K_2$ локально дискриминируется классом $\K_1$.
\end{enumerate}
\end{lemma}

\proof Докажем сперва, что из п.2 следует п.1. Отметим, что класс абелевых групп и любой его подкласс нетеровы по уравнениям. Пусть $S$ -- конечная диаграмма и $\neg \varphi_s$ не принадлежит $\Tha(\K_2)$ и $A$ группа из $\K_2$. Тогда $\neg \varphi_s \notin \Tha(A')$, для некоторой конечно порожденной подгруппы $A'$ из $A$. Так как группа $A'$ дискриминируется классом $\K_1$, то для диаграммы $S$ существует в некоторой группе $B$ из класса $\K_1$ элементы на которых выполнена диаграмм $S$, а потому $\neg \varphi_s \notin \Tha(\K_1)$, и следовательно, $\Tha(\K_1) \subseteq \Tha(\K_2)$.

Обратно, пусть $\Tha(\K_1) \subseteq \Tha(\K_2)$, $B$ -- группа из класса $\K_2$ и $S \in FD(B)$ -- конечная диаграмма, которая реализуется на элементах $b_1, \ldots, b_n \in B$. Пусть $B_0$ подгруппа в группе $B$ порожденная элементами $b_1, \ldots, b_n$. Так как $B$ -- абелева группа, то подгруппу $B_0$ можно задать конечным числом соотношений, то есть группа $B_0$ имеет представление:
 $$B_0 = \langle b_1, \ldots, b_n | r_1(\overline{b}) = 0, \ldots, r_k(\overline{b}) = 0 \rangle.$$

Расширим конечную диаграмму $S$ до конечной диаграммы $S' = S \cup \{r_1(\overline{b}) = 0, \ldots, r_k(\overline{b}) = 0\}$. Ясно, что конечная диаграмма $S'$ так же реализуется в группе $B$. Построим по конечной диаграмме $S'$ универсальную формулу $\neg\varphi_{s'}$. Ясно, что эта формула $\neg\varphi_{s'} \notin \Tha(\K_2)$, следовательно $\neg\varphi_{s'} \notin \Tha(\K_1)$. Поэтому существует группа $A \in \K_1$ такая, что конечная диаграмма $S'$ реализуется на элементах $a_1, \ldots, a_n$ в группе $A$, и существует такой гомоморфизм $\varphi: B \rightarrow A$, что $\varphi(b_i) = a_i$, $i = 1, \ldots, n$. Следовательно, класс $\K_2$ локально дискриминируется классом $\K_1$. $\square$

Сформулируем хорошо известный результат

\begin{theorem}\label{th:AbelUnivEquiv}
Любые две абелевы группы без кручения универсально эквивалентны.
\end{theorem}

Для удобства читателя мы приведем доказательство этой теоремы, которое использует основные понятия и результаты выше. Так как конечно порожденные подгруппы абелевой группы без кручения $A$ есть группы типа $\Z^n$ то в силу леммы \ref{lemma:UnivEquivForClass} достаточно доказать следующий результат.

\begin{lemma}\label{lemma:UnivEquivZk}
Группа $\Z^k$ универсально эквивалентна группе $\Z$.
\end{lemma}

\proof По лемме \ref{lemma:UnivEquiv} для универсальной эквивалентности, достаточно показать, что группа $\Z^k$ дискриминируется группой $\Z$, и группа $\Z$ дискриминируется группой $\Z^k$. То, что группа $\Z$ дискриминируется группой $\Z^k$ очевидно, так как $\Z < \Z^k$. 

Покажем, что группа $\Z^k$ дискриминируется группой $\Z$. Для этого, по определению, нужно показать, что для любого количества нетривиальных элементов $h_1, \ldots, h_n \in \Z^k$ существует гомоморфизм $\varphi : \Z^k \rightarrow \Z$, такой, что образы этих элементов не равны 0. Пусть элементы $h_1, \ldots, h_n$ имеют вид:
$$\begin{array}{c}
 h_1 = (\alpha_{11}, \ldots, \alpha_{1k}), \\
 \ldots \\
 h_n = (\alpha_{n1}, \ldots, \alpha_{nk}),
 \end{array} $$
где $\alpha_{ij} \in \Z$. Рассмотрим множество гомоморфизмов, проиндексированных целочисленными векторами длинны $k$, $\varphi_{m_1,\ldots,m_k} : \Z^k \rightarrow \Z$, действующих следующим образом:
$$\varphi_{m_1,\ldots,m_k} (\alpha_1, \ldots, \alpha_k) = m_1 \alpha_1 + m_2 \alpha_2 + \ldots + m_k \alpha_k.$$
Заметим, что если вектор $\overline{\alpha}$ задает элемент группы $\Z^k$, а вектор $\overline{m}$ задает индекс гомоморфизма, то $\varphi_{\overline{m}}(\overline{a})$ представляет скалярное произведение этих двух векторов $\overline{m} \cdot \overline{a}$. Для двух ненулевых векторов скалярное произведение не равно нулю тогда и только тогда, когда они не перпендикулярны. Рассмотрим вектор $\overline{h_1}$, в пространстве $\Z^k$ ему соответствует перпендикулярная ему гиперплоскость размерности $k-1$, аналогично и остальным векторам будут соответствовать гиперплоскости. Всего таких гиперплоскостей конечное число, а следовательно они не могут покрывать все пространство $\Z^k$, поэтому обязательно найдется вектор $\overline{m}$, который не принадлежит ни одной из этих гиперплоскостей, следовательно и гомоморфизм $\varphi_{\overline{m}}$, который будет переводит все элементы $h_1,\ldots, h_n$ в ненулевые. Следовательно, группа $\Z^k$ дискриминируется группой $\Z$. Таким образом, мы доказали, что $\Z^k \equiv_{\forall} \Z$. $\square$



\subsection{Несколько фактов об аксиоматизируемых классах}

Пусть $\L$ -- язык, $\K$ -- класс $\L$-систем. Класс $\K$ называется универсальным классом, если он аксиоматизируем и это можно сделать с помощью универсальных формул языка $\L$.

\begin{proposition}\label{prop:AxiomClass}
Класс $\K$ аксиоматизируем тогда и только тогда, когда:
\begin{enumerate}
\item Если $\mathcal{M} \in \K$ и $\mathcal{N} \equiv \mathcal{M}$ тогда $\mathcal{N} \in \K$;
\item Если $\mathcal{M}_i \in \K,$ $i \in I$, $D$ -- ультрафильтр над множеством $I$. Пусть $\overline{\mathcal{M}}$ -- ультра произведение $\mathcal{M}_i$, тогда $\overline{\mathcal{M}} \in K$.
\end{enumerate}
\end{proposition}

Следующее предложение представляет известный факт из теории моделей, доказательство можно найти в \cite{DM} (Теорема 2.3.9).
\begin{proposition}\label{prop:AxiomClass2}
Класс $\K$ универсально аксиоматизируем тогда и только тогда, когда выполнены следующих два условия:
\begin{enumerate}
\item Класс $\K$ аксиоматизируем;
\item Класс $\K$ замкнут относительно операции взятия $\L$-подсистем.
\end{enumerate}
\end{proposition}



\subsection{Элементарные инварианты для абелевых групп}\label{sec:UnivInvariants}

Введем понятие элементарных инвариантов, следуя книге Ю.Л.~Ершова \cite{Ershov}.

Пусть $A$ -- абелева группа, определим подгруппу $pA = \{ pa | a \in A\}$. Для группы $A$ с помощью индукции по $k$ определяется серия подгрупп $p^k A$. Нетрудно заметить, что $p^{k+1} A \subseteq p^k A$, и $p^k A \Big/ p^{k+1} A$ есть группа периода $p$, следовательно, на ней определена структура векторного пространства над полем $F_p$ из $p$ элементов. Таким образом, определим размерность $p^k A \Big/ p^{k+1} A$ как размерность векторного пространства $p^k A \Big/ p^{k+1} A$ над полем $F_p$.

Для абелевой группы $A$ введем первую серию инвариантов $\alpha_{p,k}(A)$, где $p$ -- простое число, а $k$ -- натуральное число, значение которых будут в расширенной аддитивной полугруппе $\N \cup \{\infty \} = \N^*$, в которой сложение в $\N$ дополнено следующими формулами: $n + \infty = \infty + n = \infty$, $\infty + \infty = \infty$.

Определим инвариант $\alpha_{p,k}(A)$ для группы $A$ таким образом:
$$\alpha_{p,k}(A) = \left\lbrace 
\begin{array}{l}  
\mathrm{dim} \left( p^{k-1}A \Big/ p^k A \right), \text{ если эта размерность конечна;} \\ 
\infty, \text{ в противном случае.}
\end{array} 
\right.$$

Введем вторую серию инвариантов $\beta_{p,k}(A)$. Ключевую роль здесь играет понятие $p$-слоя $A[p] = \{a \in A | \ pa = 0\}$. Другими словами, $p$-слой состоит из элементов порядка $p$ и нулевого элемента группы $A$. Как и выше, на $A[p]$ определяется структура векторного пространства над полем $F_p$ и понятие размерности $A[p]$ над $F_p$.

$$\beta_{p,k}(A) = \left\lbrace 
\begin{array}{l}  
\mathrm{dim} \left( (p^{k-1}A) [p] \Big/ (p^k A) [p] \right), \text{ если эта размерность конечна;} \\ 
\infty, \text{ в противном случае.}
\end{array} 
\right.$$

Введем третью серию инвариантов $\gamma_{p,k}(A)$

$$\gamma_{p,k}(A) = \left\lbrace 
\begin{array}{l}  
\mathrm{dim} \left( (p^{k-1} A) [p] \right), \text{ если эта размерность конечна;} \\ 
\infty, \text{ в противном случае.}
\end{array} 
\right.$$

Следующая лемма играет ключевую роль в доказательстве основных результатов статьи.

\begin{lemma}\label{lemma:gamma}
Пусть $A$ -- абелева группа, $\alpha_{p,k}, \beta_{p,k}, \gamma_{p,k}$ -- элементарные инварианты. Тогда:
\begin{enumerate}
\item Существует универсальная формула $\theta_{p,k,m}$ языка $L$ такая, что 
$$A \models \theta_{p,k,m} \Leftrightarrow \gamma_{p,k}(A) \leq m.$$
\item Условия $\alpha_{p,k}(A) = m$ и $\beta_{p,k}(A) = m$, где $m \in \N$ не определяются универсальными формулами;
\item Неравенства $\alpha_{p,k}(A) \leq m$ и $\beta_{p,k}(A) \leq m$, где $m \in \N$ не определяются универсальными формулами;
\end{enumerate}
\end{lemma}
\proof Докажем первый пункт леммы. Пусть $A$ --- абелева группа, и элементарный инвариант $\gamma_{p,k}(A) = m \in \N$. Тогда, по определению, $m = \gamma_{p,k} (A) = \dim_{F_p} p^{k-1} A[p]$. Обозначим $V = p^{k-1} A[p]$ --- векторное пространство над полем из $p$ элементов. Нетрудно заметить, что условие $\dim V \leq m$ эквивалентно универсальной формуле: 
$$\forall v_1, \ldots, v_{m+1} \in V \ \bigvee_{\alpha_i \in F_p} \alpha_1v_1 + \ldots + \alpha_{m+1}v_{m+1} = 0.$$
Наконец, учитывая, что $V = p^{k-1}A[p]$, запишем формулу $\theta(p,k,m)$:

$$
\theta(p,k,m) = \forall g_1, \ldots, g_{m+1} \ (\bigwedge_i p^{k}g_{i} = 0) \rightarrow  \bigvee_{\alpha_i \in F_p} \alpha_1g_1 + \ldots + \alpha_{m+1}g_{m+1} = 0.
$$

Перейдем к доказательству второго и третьего пункта леммы. Пусть условие $\alpha_{p,k}(A) = m$ задается некоторой универсальной формулой. Тогда рассмотрим абелеву группу $\Q^m$ и в ней подгруппу $\Z^m$. Заметим, что $\alpha_{p,k}(\Q^m) = 0$. Поскольку условие $\alpha_{p,k}(\Q^m) = 0$ задано универсальной формулой, то эта же формула выполняется и на подгруппе $\Z^m$, то есть $\alpha_{p,k}(\Z^m) = 0$. Но $\alpha_{p,k}(\Z^m) = m$. Противоречие. Аналогично доказывается невозможность представить условие $\alpha_{p,k}(A) \leq m$ в виде универсальной формулы.

Покажем, что условие $\beta_{p,k}(A) = m$ не может быть эквивалентно универсальной формуле. От противного, пусть такая формула существует.  Обозначим $A = C^{m}(p^{\infty})$. Заметим, что $C^{m}(p^k)$ подгруппа в $A$. Тогда, из-за универсальной определимости инварианта $\beta_{p,k}$, так как $\beta_{p,k} (A) = 0$, следует, что $\beta_{p,k} (C^{m}(p^k)) = 0$, но $\beta_{p,k} (C^{m}(p^k)) = m$ --- противоречие. 
Аналогично доказывается невозможность представить условие $\beta_{p,k}(A) \leq m$ в виде универсальной формулы. $\square$



И, наконец, введем инвариант $\delta(A)$.
$$\delta(A) = \left\lbrace 
\begin{array}{l}  
0, \text{ если } \exists m \in \N \ ma = 0 \text{ для всех } a \in A; \\ 
1, \text{ в противном случае.}
\end{array} 
\right.$$


\subsection{Классы $\A_p$}

Пусть $\A$ -- класс абелевых групп в сигнатуре $\L = \{+, -, 0\}.$ Основной целью данной статьи является классификация всех универсальных классов из $\A$. Для этого мы начинаем с классификации подклассов $\A_p$, где $p \in \P \cup \{0\}$, а $\P$ -- множество простых чисел. Подклассы $\A_p$ вводятся следующим образом:
$$\begin{array}{c}

\A_0 = \{A |\text{ если } A \text{ -- группа без кручения, либо } A = \{0\} \}, \\
\A_p = \{A | \ T(A) = T_p(A) \neq 0 \}, \ p \neq 0. \\
\end{array}$$ 

Далее для удобства множество $\P \cup \{0\}$ обозначим за $\P^*.$


\section{Универсальные инварианты для абелевых групп}

В данном параграфе мы определим для абелевой группы $A$ универсальные инварианты $\ui(A)$ с целью доказательства следующего результата: пусть $A$ и $B$ -- абелевы группы тогда $A$ и $B$ универсально эквивалентны тогда и только тогда, когда $\ui(A) = \ui(B)$.

Сузим задачу и будем считать, что группы $A$ и $B$ принадлежат классу $\A_p$, где $p \in \N^*$. Если $p \neq 0$, то инвариантом $\ui_p(A)$ для группы $A \in \A_p$ будет вектор:
$$\ui_p(A) = (\delta(A), \gamma_{p,1}(A), \gamma_{p,2}(A), \gamma_{p,3}(A), \ldots),$$
где $\delta \in \{0,1\}$, $\gamma_{p,k}(A)$ -- элементарный инвариант $\gamma_{p,k}$ для группы $A$.

Если $p = 0$, то $A$ -- абелева группа без кручения либо $A = \{0\}$. Тогда инвариант $\ui_0(A) = \delta(A)$.


Пусть $A$ -- абелева группа, определим универсальный инвариант $\ui(A)$ следующим образом:
$$\ui(A) = (\ui_{0}(A), \ui_{2}(A), \ui_{3}(A),\ui_{5}(A),\ldots, \ui_{p_i}(A), \ldots),$$
где $p_i$ -- простые числа.


\begin{theorem}\label{th:UnivEquivOfGroups}
\begin{enumerate}
\item Пусть $A$ и $B$ -- группы из $\A_p$. Тогда $A$ и $B$ универсально эквивалентны тогда и только тогда, когда $\ui_p(A) = \ui_p(B)$.
\item Две абелевы группы $A$ и $B$ универсально эквивалентны тогда и только тогда, когда $\ui(A) = \ui(B)$.
\end{enumerate}
\end{theorem}


\proof Так как, по определению, значение инварианта $\ui(C)$ определяется значениями инвариантов $\ui_p(C)$, то теорему достаточно доказать только для групп из класса $\A_p$. Если $p=0$, то в этом случае доказательство непосредственно следует из теоремы \ref{th:AbelUnivEquiv} и определения универсальных инвариантов для групп $A$ и $B$.

Пусть $p \neq 0$, $A, B \in \A_p$ и $A \equiv_\forall B$. Если $A$ без кручения и $B$ без кручения, то $\ui(A) = \ui(B)$ и результат следует из теоремы \ref{th:AbelUnivEquiv}.

Пусть $T_p(A) \neq 0$ и $T_p(B) \neq 0$. Если $\Tha(T_p(A)) \neq \Tha(T_p(B))$, то $FS(T_p(A)) \neq FS(T_p(B))$ по лемме \ref{lemma:UnivEquivFS}. Пусть $C$ -- конечная $p$-группа такая, что $C \in FS(T_p(A))$ и $C \notin FS(T_p(B))$ и $\varphi_c$ -- экзистенциальная формула, построенная по группе $C$. Тогда в $A$ выполняется универсальная формула $\neg \varphi_c$, а в группе $B$ формула $\neg \varphi_c$ не выполняется, а поэтому $\Tha(A) \neq \Tha(B)$.

Далее, если $\Tha(T_p(A)) = \Tha(T_p(B))$, но $\Tha(A) \neq \Tha(B)$, то $\delta(A) = 0$ и $\delta(B) = 1$ или $\delta(A) = 1$ и $\delta(B) = 0$. В любом случае $\ui(A) \neq \ui(B)$. 

Тогда если $\Tha(A) = \Tha(B)$, то по лемме \ref{lemma:UnivEquivFS} $FS(A) = FS(B)$ и из определения $\ui_p(A)$ для $p$-групп следует, что $\ui_p(A) = \ui_p(B)$. Если $\ui_p(A) \neq \ui_p(B)$ для $p$-групп $A$ и $B$, то снова из определения универсальных инвариантов следует, что $FS(A) \neq FS(B)$. Теорема доказана. $\square$


\section{Универсальные инварианты для универсальных классов абелевых групп}

Этот параграф посвящен классификации универсальных классов абелевых групп. Центральное место в описании этих классов занимает понятие универсального инварианта $\ui$ для класса абелевых групп. 

Описание универсальных классов в общем случае сводится к описанию универсальных классов внутри класса $\A_p$. Для этого, определим примарный универсальный инвариант $\ui_p$ для произвольного универсального класса $\K$ следующим образом:
$$\begin{array}{c}
\ui_p(\K) = (\max\limits_{A \in \K}(\delta(A)),  \max\limits_{A \in \K}(\gamma_{p,1}(A)), \max\limits_{A \in \K}(\gamma_{p,2}(A)), \ldots), \text{ если } p \neq 0; \\
\ui_0(\K) = \max\limits_{A \in \K}(\delta(A)); \\
\end{array}$$
где $p$ -- простое число.

Универсальным инвариантом $\ui(\K)$ для универсального класса $\K$ будем называть бесконечный вектор элементами которого являются все примарные универсальные инварианты $\ui_p(\K)$:
$$\ui(\K) = (\ui_0(\K), \ui_2(\K), \ui_3(\K), \ldots, \ui_{p_i}(\K), \ldots ),$$
где $p_i$ -- простые числа.

\begin{theorem}\label{th:UnivClasses}
Пусть $\K_1$ и $\K_2$ -- два универсально аксиоматизируемых класса абелевых групп. Тогда $\Tha(\K_1) = \Tha(\K_2)$ тогда и только тогда, когда $\ui(\K_1) = \ui(\K_2)$.
\end{theorem}
\note{Отредактировать доказательство}
\proof Так как классы $\K_1$ и $\K_2$ нетеровы по уравнениям, то по лемме \ref{lemma:UnivEquivForClass} $\K_1 \equiv_\forall \K_2$ тогда и только тогда, когда класс $\K_1$ локально дискриминируется классом $\K_2$ и наоборот класс $\K_2$ локально дискриминируется классом $\K_1$. Следовательно, если $\K_1 \equiv_\forall \K_2$, то $FS(\K_1) = FS(\K_2)$, а потому все компоненты $\ui_p(\K_1)$ совпадают с соответствующими компонентами $\ui_p(\K_2)$, за исключением быть может первой компоненты. Пусть $\max\limits_{A \in \K_1}(\delta(A)) \neq \max\limits_{A \in \K_2} (\delta(A))$, тогда один класс состоит из ограниченных в совокупности числом $m$ групп (то есть имеет место $mx = 0$ $\forall x \in A$ и для $\forall A \in \K_1$), а для второго класса $\K_2$ $\ui_0(\K_2) = 1$, что противоречит условию, что $\K_1 \equiv_\forall \K_2$. Следовательно, $\ui(\K_1) = \ui(\K_2)$.

Наоборот, если $\ui(\K_1) = \ui(\K_2)$, то $FS(\K_1) = FS(\K_2)$. Если $\ui_0(\K_1) = \ui_0(\K_2) = 0$, то классы $\K_1$ и $\K_2$ локально дискриминируются друг друга, и следовательно, $\K_1 \equiv_\forall \K_2$. Если $\ui_0(\K_1) = \ui_0(\K_2) = 1$, то по лемме \ref{lemma:delta1UltraProduct} {\color{red}???} классы $\K_1$ и $\K_2$ содержат $\Z$, а следовательно по лемме {\color{red}???} и $\Z^m$ для всех $m \in \N$. Поэтому классы $\K_1$ и $\K_2$ взаимно локально дискриминируют друг друга, и следовательно, по лемме \ref{lemma:UnivEquivForClass} $\K_1 \equiv \K_2$. Теорема доказана. $\square$





\section{Канонические группы}\label{sec:CannonicalGroups}

Пусть $\K$ произвольный универсальный класс абелевых групп. Цель данного параграфа по классу $\K$ построить единственную каноническую группу этого класса $C(\K)$ с таким свойством, что две канонические группы $C(\K_1)$ и $C(\K_2)$ изоморфны тогда и только тогда, когда классы $\K_1$ и $\K_2$ совпадают.

Заметим, что пересечение универсальных классов $\K \cap \A_p$ также является универсальным классом, и $\K = \bigcup\limits_{p \in \P} \K_p$. Мы стартуем с построения канонической группы для универсального класса $\K_p$, а построение канонической группы для универсального класса $\K$ в общем случае будет проведено в конце данного параграфа.

Пусть задан допустимый универсальный инвариант 
$$\ui_p = (\delta, \gamma_{p,1}, \gamma_{p,2}, \ldots).$$
Напомним, что инвариант называется допустимым, если $\delta \in \{0,1\}$, $\gamma_{p,k} \in \N^* = \N \cup \{\infty\}$ и для любого $k$ выполнено $\gamma_{p,k} \geq \gamma_{p, k+1}$.

По допустимому инварианту $\ui_p$ построим каноническую группу $C$ из класса $\A_p$. Для этого, последовательность значений инварианта $\gamma_{p,k}$ разделим на следующих три непересекающихся интервала (некоторые из интервалов могут быть пустым). Это разделение будет определять три числа: $a$, $b \in \N \cup \{0\}$ и $l \in \N^* \cup \{0\}$. Введем эти параметры следующим определением:
\begin{enumerate}
\item Параметр $l \in \N^* \cup \{0\}$ определяем как $l = \lim\limits_{k \rightarrow \infty} \gamma_{p,k}$;
\item Если $l \in \{\infty, 0\}$, то $a = b = 0$;
\item Если $l \in \N$, то параметры $a$ и $b$ определяются следующим образом:
\begin{itemize}
\item параметр $b = \min\limits_{i} \{ i | \ \gamma_{p,i} = l\}$, в этом случае $\gamma_{p,k} = l$ для всех $k \geq b$;
\item параметр $a = \max\limits_{i} \{i | \ \gamma_{p,i} = \infty\}$, если $\{i | \ \gamma_{p,i} = \infty\} = \emptyset$, то $a = 0$.
\end{itemize}
\end{enumerate}

Опираясь на значения этих трех параметров $a, b$ и $l$, построим группу $C$ следующим образом:
$$C = C^\infty(p^a) \oplus T \oplus C^l(p^\infty) \oplus B ,$$
где группа $T = \bigoplus\limits_{ a < t \leq b} C^{w_t}(p^t)$, где $w_t = \gamma_{p,t} - \gamma_{p,t+1}$, и группа $B$ либо $\Z$, при $l = 0$ и $\delta = 1$, либо $B = 0$, в остальных случаях. Отметим, что подгруппа $T_p(C) = C^\infty(p^a) \oplus T \oplus C^l(p^\infty)$ будет периодической частью группы $C$, которая в свою очередь будет делится на редуцированную подгруппу $C^\infty(p^a) \oplus T$ и делимую подгруппу $C^l(p^\infty)$ группы $C$.

Множество всех канонических групп, построенных по допустимым характеристикам $\ui_p$ обозначим через $\CG_p$.

\begin{proposition}\label{prop:UnivEnvForCannonicalGroup}
Если каноническая группа $C \in \CG_p$ построена по допустимому универсальному инварианту $\ui_p$, то $\ui_p(C) = \ui_p$.
\end{proposition}
\proof Данное предложение следует непосредственно из определения универсальных инвариантов и структуры группы $C$. $\square$

\begin{theorem}
Для групп из класса $\CG_p$ верны следующие утверждения:
\begin{enumerate}
\item Если $C_1, C_2 \in \CG_p$ пара неизоморфных групп, то $\Tha(C_1) \neq \Tha(C_2)$;
\item Для любой группы $A$ из класса $\A_p$ существует такая единственная группа $C \in \CG_p$, что $\Tha(A) = \Tha(C)$;
\item Для любого универсального класса абелевых групп $\K$ из $\A_p$ существует такая единственная группа $C \in \CG_p$, что $\Tha(\K) = \Tha(C)$.
\end{enumerate}
\end{theorem}
\proof Из предложения \ref{prop:UnivEnvForCannonicalGroup} следует, что если две группы $C_1, C_2 \in \CG_p$ неизоморфны, то существует такое натуральное число $k$, что $\gamma_{p,k}(C_1) \neq \gamma_{p,k}(C_2)$. Значения этих универсальных инвариантов определяется универсальной формулой (по лемме \ref{lemma:gamma}). Отсюда и следует доказательство пункта 1 теоремы.

Для доказательсвта пункта 2 вычислим для группы $A$ универсальный инвариант $\ui_p(A)$. По данному инварианту построим каноническую группу $C$. По предложению \ref{prop:UnivEnvForCannonicalGroup} $\ui_p(C) = \ui_p(A)$. Следовательно, по теореме \ref{th:UnivEquivOfGroups}, группы $A$ и $C$ универсально эквивалентны.

\note{Доказать п.3}

$\square$

\note{Вставить определение класса $\CG$}

\begin{theorem}\label{th:CannonicalGroups}
Для групп из класса $\CG$ верны следующие утверждения:
\begin{enumerate}
\item Если $C_1, C_2 \in \CG$ пара неизоморфных групп, то $\Tha(C_1) \neq \Tha(C_2)$;
\item Для любой абелевой группы $A$ существует такая единственная группа $C \in \CG$, что $\Tha(A) = \Tha(C)$;
\item Для любого универсального класса абелевых групп $\K$ существует такая единственная группа $C \in \CG$, что $\Tha(\K) = \Tha(C).$
\end{enumerate}
\end{theorem}
\note{Вставить доказательство}



\begin{proposition}
\note{Проверить нужно ли данное предложение}
Пусть $\K$ -- универсально аксиоматизируемый класс абелевых групп, $p$ -- простое число и для любого $n \in \N$ группа $C(p^n) \in \K$. Тогда:
\begin{enumerate}
\item $\Z \in \K$;
\item $C(p^\infty) \in \K$.  
\end{enumerate}
\end{proposition}

\proof Пусть $F$ -- фильтр Фреше над множеством натуральных чисел. Рассмотрим фильтрованное произведение групп $C(p^k)$ по фильтру $F$ и обозначим его $A$:
$$A = \prod_{k \in \N} C(p^k) \Big/ F.$$ 
По предложению \ref{prop:AxiomClass} группа $A$ принадлежит классу $\K$. 
Пусть каждая группа $C(p^k)$ порождается элементом $a_i$, тогда, если взять элемент $a = (a_1, a_2, \ldots) \in A$, то он порождает группу $\Z$. Таким образом, группа $\Z$ является подгруппой группы $A$, следовательно по предложению \ref{prop:AxiomClass} группа $\Z \in \K$.

Далее покажем, что группа $C(p^\infty)$ является подгруппой в группе $A$. Для этого, отметим, что группа $C(p^\infty)$ имеет следующее представление:
$$C(p^\infty) = \langle g_1, g_2, g_3, \ldots | \ pg_1 = 1, pg_2 = g_1, pg_3 = g_2, pg_4 = g_3, \ldots \rangle.$$
Выберем произвольный элемент $a \in A$, пусть $a$ имеет вид $a = (a_1, a_2, a_3, \ldots )$. Возьмем в качестве $b_1, b_2, b_3, \ldots$ следующие элементы группы $A$:
$$\begin{array}{ccl}
b_1 & = & (a_1, pa_2, p^2a_3, p^3a_4, \ldots), \\
b_2 & = & (a_1, a_2, pa_3, p^2a_4, \ldots), \\
b_3 & = & (a_1, a_2, a_3, pa_4, \ldots), \\
&  & \ldots \\
\end{array}$$
Заметим, что $pb_k = b_{k-1}$ для всех $k$, следовательно, указанные элементы $b_1, b_2, b_3, \ldots$ порождают группу $C(p^\infty)$, следовательно $C(p^\infty) < A$, и по предложению \ref{prop:AxiomClass2} группа $C(p^\infty) \in \K$. $\square$ 


\section{Технические леммы}\label{sec:lemmas}

В данном параграфе мы докажем технические леммы, которые нам понадобятся для доказательства теорем. 


\begin{lemma}
Пусть $\K$ -- произвольное множество конечных $p$-групп, $\K = \{A_i | \ i \in I\}$, и $D$ -- неглавный ультрафильтр над $I$, $U$ -- ультрапроизведение 
$$U = \prod A_i \Big/ D.$$
Тогда если $\delta(\K) = 0$, то существует такое натуральное число $m$, что на $U$ выполнена формула $\varphi_m: \forall x \ mx = 0;$
\end{lemma}

\proof


\begin{lemma}\label{lemma:delta1UltraProduct}
Пусть $\K$ -- произвольное множество конечных $p$-групп, $\K = \{A_i | \ i \in I\}$, и $\delta(\K) = 1$. Тогда существует подмножество $I_1 \subseteq I$ и $D$ -- неглавный ультрафильтр над $I_1$, такое что для ультрапроизведения $U$
$$U = \prod_{i \in I_1} A_i \Big/ D$$
выполнены следующие условия:
\begin{enumerate}
\item Группа $\Z$ является подгруппой в $U$;
\item Если $\ui_{p,\infty}(\K) = n$, то $U$ содержит прямую сумму $n$ копий квазициклической $p$-группы $C(p^\infty)$, но не содержит прямую сумму $n+1$ копий квазициклической $p$-группы;
\item Если $\ui_{p,\infty}(\K) = \infty$, то $U$ содержит счетное количество копий квазициклической $p$-группы $C(p^\infty)$;
\end{enumerate}
\end{lemma}

\proof Покажем, что существует бесконечная цепочка групп $A_{i_1}, A_{i_2}, \ldots \in \K$ такая, что для любого натурального числа $k$ группа $A_{i_k}$ содержит группу $C(p^k)$. Условие $\delta(\K) = 1$ означает, что для любого натурального числа $m$ существует группа $A$ в классе $\K$ такая, что в ней есть элемент $a$ такой, что $ma \neq 0$. Далее, строим нашу цепочку по индукции. Предположим, что мы выбрали $k-1$ группу $A_{i_1}, \ldots, A_{i_{k-1}}$. Выбираем максимальный порядок элементов всех этих групп, обозначим его $l$, заметим, что $l > k$. На $k$-ом шаге выбираем в классе $\K$ такую группу $A_{i_k}$ в которой есть элемент $a$ такой, что $p^{l} a \neq 0$. Группа $A_{i_k}$ не может совпадать с ранее выбранными группами так как содержит в себе элемент порядка больше чем $p^l$, и, следовательно, $C(p^{l+1})$ подгруппа в $A_{i_k}$. Из неравенства $k < l+1$, имеем, что $C(p^k)$ подгруппа в $A_{i_k}$.

В качестве множества $I_1$ возьмем множество индексов выбранной цепочки групп $\{i_1, i_2, \ldots\}$.

Выберем в каждой группе $A_{i_k}$ элемент $a_{i_k}$ который порождает циклическую подгруппу $C(p^k) = \langle a_{i_k}\rangle$.

Для доказательства пункта 1, рассмотрим элемент $a = (a_{i_1}, a_{i_2}, \ldots) \in U$. Элемент $a$ имеет бесконечный порядок в группе $U$, следовательно он порождает подгруппу изоморфную группе $\Z$ в $U$. 

Для доказательства пункта 2, выберем следующую серию элементов:
$$\begin{array}{l}
b_1 = (a_1, p a_2, p^2 a_3, p^3 a_4, \ldots), \\
b_2 = (a_1, a_2, p a_3, p^2 a_4, \ldots), \\
b_3 = (a_1, a_2, a_3, p a_4, \ldots), \\
\ldots
\end{array}$$

Нетрудно проверить, что элементы $b_1, b_2, \ldots$ удовлетворяют следующим соотношениям в группе $U$: $pb_1 = 1$, $pb_2 = b_1$, $pb_3 = b_2, \ldots$. Следовательно эти элементы пораждают подгруппу изоморфную $C(p^\infty)$.

$\square$


\begin{lemma}
Пусть $\K$ произвольное множество конечных $p$-групп и $ucl(\K)$ --- универсальное замыкание $\K$. Тогда класс $ucl(\K)$ аксиоматизируется следующим набором формул:
\begin{enumerate}
\item Аксиомы абелевых групп.
\item Аксиомы класса $\A_p$: $\{ \forall x \ qx = 0 \rightarrow x = 0 | (p,q = 1) \}$.
\item Если $\delta(\K) = 0$, то формулы:
\begin{enumerate}
\item $\varphi_m: \forall x \ mx = 0;$ где $m$ --- такое натуральное число, для которого формула $\varphi_{m-1}$ не выполнена на какой-либо группе из $\K$, но $\K \models \varphi_m$. 
\item Конечное множество формул $\{\theta_{p,k,m} |  m = \ui_{p,k}(\K) > 0 \}$.
\end{enumerate}
\item Если $\delta(\K) = 1$, то:
\begin{enumerate}
\item Если $\ui_{p, \infty}(\K) \neq \infty$, то множество формул $\{ \theta_{p,k,m} | k \in \N, m = \ui_{p,k}(\K)\}$.
\item В случае, когда $\ui_{p, \infty}(\K) = \infty$ множество аксиом состоит только из формул описанных в пунктах 1 и 2 леммы.
\end{enumerate}
\end{enumerate}
\end{lemma}








\begin{thebibliography}{}

\bibitem{Szm} Szmielew W. \textit{Elementary properties of Abelian groups.} — Fundamenta Mathematica. —
1955. — v. 41. — p. 203–271.

\bibitem{DM} D.Marker, \textit{Model Theory: An Introduction}, // Springer, Series: Graduate Texts in Mathematics, Vol. 217, 2002, 342 p.

\bibitem{Ershov} Ершов Ю.Л. \textit{Проблемы разрешимости и конструктивные модели.} --- М.: Наука, 1980.

\bibitem{DMR1} E. Daniyarova, A. Miasnikov, V. Remeslennikov, \textit{Unification theorems in algebraic geometry}, // Algebra and Discrete Mathematics, 1 (2008), 80--111, arXiv: 0808.2522.


\bibitem{DMR2} \note{Проверить ссылку} Э. Ю. Даниярова, А. Г. Мясников, В. Н. Ремесленников, \textit{Алгебраическая геометрия над алгебраическими системами. II. Основания}, // Фундамент. и прикл. матем., 17:1 (2012), 65--106, arXiv: 1002.3562; E. Yu. Daniyarova, A. G. Myasnikov, V. N. Remeslennikov, \textit{Algebraic geometry over algebraic structures. II. Foundations}, // J. Math. Sci., 185:3 (2012), 389--416.

\end{thebibliography}


\end{document}